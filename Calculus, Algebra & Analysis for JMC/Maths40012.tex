
\documentclass[12pt]{report}
\usepackage[thinc]{esdiff} % for typesettign derivatives
\usepackage{amsthm} % provides an enhanced version of LaTex's \newtheorem command
\usepackage{mdframed} % framed environments that can split at page boundaries
\usepackage{enumitem} % bulletin points or other means of listing things
\usepackage{amssymb} % for AMS symbols
\usepackage{amsmath} % so as to use align

\theoremstyle{definition}
\mdfdefinestyle{defEnv}{%
  hidealllines=false,
  nobreak=true,
  innertopmargin=-1ex,
}

\pagestyle{headings}
\author{Marie Amellie}
\title{Calculus, Algebra, and Analysis for JMC}
\begin{document}
\maketitle
\tableofcontents

\chapter{Group theory}

Study of the simplest algebraic structure on a set.

\section{Binary operations and groups}

\newmdtheoremenv[style=defEnv]{theorem}{Definition}
\begin{theorem}
    \emph{Set} is a collection of distinct elements. Let $G$ be a set.
    \textbf{\emph{Binary operation on G}} is a function\[
        *: G \times G \rightarrow G \textnormal{(Closure is included)}
    \]
\end{theorem}

\newtheorem{ex}[theorem]{Example}
\begin{ex}
    \;
    
    \begin{itemize}
        \item $(\mathbb{N}, +), (\mathbb{Z}, +), (\mathbb{R}, \cdot)$
        \item $(\mathbb{N}, -)$ not a binary op. Not closed.
        \item $g, h \in G, g * h = h$
        \item Find a certain $c \in G$, define $g*h = c \;\forall g, h \in G$
    \end{itemize}
    
\end{ex}

\begin{ex}
    Cayley table: Draw a table of all the possible binary operations on a set.
    How many possible binary operations on a finite set with $n$ elements?
    In general, there are $\infty$-many biniary operatiions. In this case, there are $n^{n^2}$ possible binary operations.
    \emph{In general, $g_i * g_j \neq g_j * g_i$} (Not commutative!)
    
\end{ex}

\newmdtheoremenv[style=defEnv]{Associativity}[theorem]{Definition}
\begin{Associativity}
    A binary operation $*$ on a set $G$ is called associative if\[
        (g*h)*k = g*(h*k) \;\forall g,h,k \in G
    \]
\end{Associativity}

\begin{ex}
    \;

    \begin{itemize}
        \item $+$ on $\mathbb{N, Z, R}$? Yes
        \item $-$ on $\mathbb{R}$? No
        \item $g*h = g^{h}$ on $\mathbb{N}$? No
    \end{itemize}
    
\end{ex}

\newmdtheoremenv[style=defEnv]{commutative}[theorem]{Definition}
\begin{commutative}
    A binary operation is called commutative if \[
        \forall g, h \in G, g * h = h * g
    \]
\end{commutative}

\begin{ex}
    \;

    \begin{itemize}
        \item $+, \cdot$ on $\mathbb{N, Z, R, C}$
        \item matrix multiplication ($AB \neq BA$ in general for $A, B$ in $M(\mathbb{R}^{n})$)
        \item let $g, h \in \mathbb{R}$, $g * h = 1 + g \cdot h$: commutative but \emph{not associative}!
    \end{itemize}
    
\end{ex}

\newmdtheoremenv[style=defEnv]{identity element}[theorem]{Definition}
\begin{identity element}
    Let $(G, *)$ be a set. An element $e$ is called \emph{left identity} (respectively \emph{right identity}) if:\[
        e * g = g (\textnormal{resp.}\; g * e = g) \;\forall\; g \in G
    \]
    \underline{Caution}: There might be \emph{many} left/right identities or none.
\end{identity element}

\begin{ex}
    \;

    \begin{enumerate}
        \item let $(G,*)$ be a set with $g*h:=g$.
            Find the left/right identities.

            $\infty$-many (or equal to the number of elements)
            right identities since $h$ satisfies definition $\forall h$.
            No left identities: wanted $e * g = g = e$ by definitioin of $*$ (\emph{unless only one element}).
        \item $(G, *)$, $g*h = 1 + gh$. 
            Ex: No right/left identities.
    \end{enumerate}
\end{ex}

Idea: We want a good unique identity.
\newmdtheoremenv[style=defEnv]{unique identity}[theorem]{Theorem}
\begin{unique identity}
    let $(G, *)$ be set, such that $*$ has both a left identity $e_1$ \emph{and}
    a right identity $e_2$, then\[
        e_1 = e_2 =: e \quad \textnormal{and} \quad e \textnormal{ is unique.}
    \]
\end{unique identity}

\begin{proof}
    \;

    \begin{itemize}
            \item $e_1 = e_2$
    \[
    \Rightarrow \left\{
        \begin{array}{l}
        e_1 * g = g \Rightarrow e_1 * e_2 = e_2 \\
        g * e_2 = g \Rightarrow e_1 * e_2 = e_1
    \end{array}
\right\} \forall\,g \in G \Rightarrow\,e_1 = e_2%chktex 1
\]
            \item Unicity: Assume there exists another identity $e'$.\[
                \Rightarrow e' * g = g * e' = g
            \]\[
                e' * g = e' * e = e
            \]\[
                g * e' = e * e' = e'
            \]Therefore \[
                e = e'
            \]
    \end{itemize}
    
\end{proof}
As soon as you get one left and one right identity, you have a unique identity $e$.

\newmdtheoremenv[style=defEnv]{inverse}[theorem]{Definition}
\begin{inverse}
    let $(G,*)$ be a set. Let $g \in G$.
    An element $h \in G$ is called left (resp.\ right) inverse if\[
        h * g = e \;(\textnormal{resp. } g * h = e)
    \]
    \underline{Caution}: Again inverses might not exist, 
    there might be many, or \emph{not} the same on both sides.
\end{inverse}

\begin{ex}
    \,

    \begin{enumerate}[label = (\arabic*)]
        \item $(\mathbb{N}, \cdot)$
            1 has an inverse, otherwise \emph{no} inverse.
        \item Find a binary operation on a set of 4 elements with left/right inverses
            not the same but identity $e$.
    \end{enumerate}
    
\end{ex}

\newmdtheoremenv[style=defEnv]{equal left right inverse}[theorem]{Theorem}
\begin{equal left right inverse}
    Let $(G, *)$ be a set with associative binary operation and identity $e$.
    Then if $h_1$ is left inverse, and $h_2$ is right inverse, then\[
        h_1 = h_2 = g^{-1} \;\textnormal{ and \emph{it is unique}}
    \]
\end{equal left right inverse}

\begin{proof}
    \;

    \begin{itemize}
            \item $h_1 = h_2$

                $h_1 * g = e, g * h_2 = e$. Therefore\[
                    h_2 = e * h_2 = (h_1 * g) * h_2 = h_1 * (g * h_2) = e = h_1
                \]
            \item unicity: Assume $\exists g'^{-1}$ another inverse.\[
                    g'^{-1} = e * g'^{-1} = (g^{-1} * g) * g'^{-1} 
                    = g^{-1} * (g * g'^{-1}) = g^{-1} * e = g^{-1}
            \]
    \end{itemize}
\end{proof}

\newmdtheoremenv[style=defEnv]{Group Definition}[theorem]{(Group) Definition}
\begin{Group Definition}
    A set $(G,*)$ wth binary operatioin $*$ is called a \emph{group} if:
    \begin{enumerate}[label = (\arabic*)]
        \item $*$ is associative
        \item $\exists e \in G$ an identity $\forall g \in G$
        \item All elements $g \in G$ have an inverse $g^{-1}$
    \end{enumerate}
    \underline{Attention}: The identity and inverses are \emph{unique} by our previous results.
    
\end{Group Definition}

\begin{ex}
    \;

    \begin{itemize}
        \item $(\mathbb{Z}, +), (\mathbb{Z}_n, +)$ are groups.
        \item $(\mathbb{N}, +)$ not a group $\Rightarrow$ no inverses.
        \item $(\mathbb{C}, \cdot)$ not a group (0 has no multiplicative inverse),
            but $(\mathbb{C}^{*}, \cdot)$ is. ($\mathbb{C}^{*} = \mathbb{C}\backslash \{0\}$)
        \item $(G = \{e\}, *)$ with $e * e = e$ is a group called the \emph{trivial group}.
        \item Empty set $\varnothing$ is not a group (No identity element.)
    \end{itemize}
    
\end{ex}

\newmdtheoremenv[style=defEnv]{finite group}[theorem]{Definition}
\begin{finite group}
    Let $G$ be a group. It is called \underline{finite} if it has finitely many elements.
    
    \underline{Notation}: $|G| = n$ (number of elements)

    If $|G| = \infty$, the $G$ is called an infinite group.
\end{finite group}

\begin{ex}
    \;

    \begin{itemize}
        \item the trivial group is finite, $|G| = 1$
        \item let $G = \{1, -1, i, -i\} \subset \mathbb{C}$, with $* = \cdot$.
            Is it a group? Yes. Check associativity, identity, and inverses.
    \end{itemize}
    
\end{ex}

\newmdtheoremenv[style=defEnv]{abelian group}[theorem]{(Abelian Group) Definition}
\begin{abelian group}
    A group is called \emph{Abelian} if $*$ is commutative.
\end{abelian group}

\begin{ex}
    \;

    \begin{itemize}
            \item previous example, trvial group, $(\mathbb{Z}, +), (\mathbb{C}^{*}, \cdot)$
            \item let $GL(\mathbb{R}^{n})$ be the set of all invertible $n \times n$ matrices, $* = $ matrix multiplication.
                It is associative: $(AB)C = A(BC)$;
                It has identity: $I_n$.
                It has inverses: yes since we asked for it.
                So this is a group of matrices.
                But this is not Abelian since $AB \neq BA$.
            \item let $G$ be the set of \emph{invertible} functions with $* = \circ$, the composition of functions.
                Identity is $F(x) = x$; they are associative, invertible, but \emph{not Abelian}.
    \end{itemize}
    
\end{ex}

\section{Consequences of the axioms of group}



\chapter{Applied Mathematical Methods}

\section{Differential Equations}

\subsection{Definitions and examples}

\newmdtheoremenv[style=defEnv]{ODE}[theorem]{Definition}
\begin{ODE}
    An \emph{ordinary differential equation} (ODE) for $y(x)$ is an equation involving \underline{derivatives} of $y$.
    \begin{equation}\label{ODE:1}
        f(x, y, \frac{\mathrm{d} y}{\mathrm{d}x}, \frac{\mathrm{d}^{2} y}{\mathrm{d}x^{2}}, \ldots,
        \frac{\mathrm{d}^{n} y}{\mathrm{d}x^{n}}) = 0
        \end{equation}
    \[
        \frac{\mathrm{d}^{n} y}{\mathrm{d}x^{n}} = 
        F(x, y, \frac{\mathrm{d} y}{\mathrm{d}x}, \ldots, \frac{\mathrm{d}^{n-1} y}{\mathrm{d}x^{n-1}})
    \]
    and we seek a solution (or solutions) for $y(x)$ satisfying the equations.
    (If there are more independent variables then we have a partial differential equation (PDE).)
\end{ODE}

\newmdtheoremenv[style=defEnv]{order and power}[theorem]{Definition}
\begin{order and power}
    \;

    \emph{Order} is the order of the highest derivative present.

    \emph{Degree} is the power of the highest derivative when fractional powers have been removed.

    \emph{\textbf{Linear} differential equation} is a differential equation that is defined by a \emph{linear polynomial}
    in the unknown function and its derivative in each term of equation\eqref{ODE:1}.
\end{order and power}

\begin{ex}
    \;

    \begin{enumerate}[label = (\alph*)]
        \item \underline{Particle moving along a line} with a given force $\rightarrow x(t)$ position
            as function of time $t$.\[
                \frac{\mathrm{d}^{2} x}{\mathrm{d}t^{2}} = f\left(t, x, \frac{\mathrm{d} x}{\mathrm{d}t} \right) 
            \]e.g.\[
                \frac{\mathrm{d}^{2}x}{\mathrm{d}t^{2}} = -\omega^{2}x - 2k\frac{\mathrm{d} x}{\mathrm{d}t} 
            \]
            The first term is regarding the restoring force,
            while the second term is regarding the damping/friction.
            The function is of order 2, degree 1, and linear.

        \item \underline{Radius of curvature} of a curve

            It can be shown that \[
                R(x,y) = \frac{{\left[1 + {\left(\frac{\mathrm{d} y}{\mathrm{d}x} \right)}^{2}\right]}^{\frac{3}{2}}}
                {\frac{\mathrm{d}^{2}y}{\mathrm{d}x^{2}} }
            \]
            The function is of order 2 and degree 2.

        \item \underline{Simple growth and decay}\[
            \frac{\mathrm{d} Q}{\mathrm{d}t} = kQ
        \]
        The function is of order 1, degree 1, and linear.\ e.g. 
        \begin{enumerate}[label = (\arabic*)]
            \item $k > 0$. $Q$ as the quantity of money, and $k = (1 + \frac{r}{100})$, and $r$ being the rate of interest.
            \item $k < 0$. $Q$ as the amount of radioactive material, and $k$ as the decay rate.
        \end{enumerate}
        Hence, obviously $Q(t) = Q_0e^{kt}$ where $Q_0 = Q(0)$ at $t = 0$.

    \item \underline{Population dynamics}

        $P(t)$ as population over time and $F(t)$ as food over time, with
        \begin{equation}\label{ODE:2}
                \frac{\mathrm{d} P}{\mathrm{d}t} = aP (a>0)
            \end{equation}
            
        \[
        \frac{\mathrm{d} F}{\mathrm{d}t} = c (c>0)
        \]
        These two equations form a linear system, with both being of order 1, degree 1.
        
        So $P(t) = P_0 e^{at}, F(t)=ct + F_0$.
        Misery! Population outgrows food supply.

        Pierre Verhulst (1845) replaced $a$ in equation\eqref{ODE:2} with $(a - bP)$
        so that growth decreases as $P$ increases:
        \begin{equation}\label{ODE:3}
            \frac{\mathrm{d} P}{\mathrm{d}t} = aP - bP^{2}
        \end{equation}
        This is in fact a \emph{logistic ODE}, with order 1, degree 1, and nonlinear.

        \underline{Note}: Equation\eqref{ODE:3} is \emph{separable}. Alternatively we can note that 
        equation\eqref{ODE:3} is an example of a \textit{Bernoulli differential equation}\[
            \frac{\mathrm{d} y}{\mathrm{d}x} + F(x)y = H(x)y^{n}
        \]with $n\neq 0,1$
        Substitution on $z(x) = {(y(x))}^{1-n} \Rightarrow$ a \emph{linear} equation for 
        $z(x) \rightarrow $ solution. (See below)

    \item \underline{Predator-Prey System}

        $x(t)$ as prey and $y(t)$ as predators, we have
        \begin{equation}\label{ODE:4}
            \frac{\mathrm{d} x}{\mathrm{d}t} = ax - bxy,\quad
            \frac{\mathrm{d} y}{\mathrm{d}t} = -cy + \hat{d}xy
        \end{equation}
        \underline{Note}: Equation\eqref{ODE:4} is \emph{separable} when written in principle\[
            \frac{\mathrm{d} y}{\mathrm{d}x} = \frac{\frac{\mathrm{d} y}{\mathrm{d}t} }
            {\frac{\mathrm{d} x}{\mathrm{d}t} } \Rightarrow y(x) \Rightarrow x(t), y(t)
        \]
        This is of order 1, degree 1, and a nonlinear system.
        
    \item \underline{Combat Model System} 
        \begin{equation}\label{ODE:5}
            \frac{\mathrm{d}x}{\mathrm{d}t} = -ay, \quad
            \frac{\mathrm{d}y}{\mathrm{d}t} = -bx
        \end{equation}
        This is of order 1, degree 1, and linear system.

        \underline{Note}: Again equation\eqref{ODE:5} is \emph{separable} when written as
        $\frac{\mathrm{d} y}{\mathrm{d}x} = \frac{bx}{ay} \Rightarrow y(x) \Rightarrow x(t), y(t)$
    \end{enumerate}
    
\end{ex}

\subsection{General Statement}




\chapter{linear algebra}

\chapter{Analysis}


\end{document}
